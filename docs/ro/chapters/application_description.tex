\chapter{Descrierea aplicației}
...
    \section{Problema adresată}
    \label{Problema adresată}
    Foarte puține platforme de social media au ca prioritate siguranța și opiniile utilizatorilor.
    
    \section{Soluția propusă}
    \label{Soluția propusă}
    FriendHub se axează pe siguranța și protejarea datelor utilizatorilor făcând funcționalitățile ce le pot pune în pericol datele ca implicit dezactivate. Există ghiduri și atenționări care îi ajută pe cei doritori să se bucure de noile funcții ale aplicației.
    
    Prin opțiunile de raportare erori sau cerere funcționalități direct din aplicație, utilizatorii pot sa-și exprime opiniile cât mai ușor pentru a fi ascultate de dezvoltatori.
    \section{Funcționalitățile aplicației}
    \label{Funcționalitățile aplicației}
    ...
        \subsection{Detectare facială și etichetare automată}
        \label{Funcționalitățile aplicației 1}
        Persoanele regăsite într-o fotografie postată de un utilizator pot fi automat etichetate. Această funcționalitate este opțională din cauza problemelor de privacy ce le poate cauza. Atât autorul cât și persoanele din imagine trebuie să accepte această opțiune. În cazul în care o persoană din poză optează pentru detectarea facială, după postarea acelei poze, va primi o notificare în care este întrebată dacă acceptă etichetarea automată. În cazul refuzului, autorul nu este notificat. Scopul acestei funcționalități este de a permite utilizatorilor să-și lărgească cât mai facil cercul de prieteni.
        \subsection{2}
        \label{Funcționalitățile aplicației 2}
        ...
    \section{Soluții similare}
        \label{Soluții similare}
        ...
        \subsection{Meta - Facebook}
        \label{Soluții similare Meta - Facebook}
        ...
        \subsection{Twitter}
        \label{Soluții similare Twitter}
        ...
        \subsection{Reddit}
        \label{Soluții similare Reddit}
        ...
        \subsection{Discord}
        \label{Soluții similare Discord}
        ...
\newpage